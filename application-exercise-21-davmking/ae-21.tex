% Options for packages loaded elsewhere
\PassOptionsToPackage{unicode}{hyperref}
\PassOptionsToPackage{hyphens}{url}
%
\documentclass[
]{article}
\title{AE 21: Multiple Regression, cont}
\author{Dav King}
\date{3/29/2022}

\usepackage{amsmath,amssymb}
\usepackage{lmodern}
\usepackage{iftex}
\ifPDFTeX
  \usepackage[T1]{fontenc}
  \usepackage[utf8]{inputenc}
  \usepackage{textcomp} % provide euro and other symbols
\else % if luatex or xetex
  \usepackage{unicode-math}
  \defaultfontfeatures{Scale=MatchLowercase}
  \defaultfontfeatures[\rmfamily]{Ligatures=TeX,Scale=1}
\fi
% Use upquote if available, for straight quotes in verbatim environments
\IfFileExists{upquote.sty}{\usepackage{upquote}}{}
\IfFileExists{microtype.sty}{% use microtype if available
  \usepackage[]{microtype}
  \UseMicrotypeSet[protrusion]{basicmath} % disable protrusion for tt fonts
}{}
\makeatletter
\@ifundefined{KOMAClassName}{% if non-KOMA class
  \IfFileExists{parskip.sty}{%
    \usepackage{parskip}
  }{% else
    \setlength{\parindent}{0pt}
    \setlength{\parskip}{6pt plus 2pt minus 1pt}}
}{% if KOMA class
  \KOMAoptions{parskip=half}}
\makeatother
\usepackage{xcolor}
\IfFileExists{xurl.sty}{\usepackage{xurl}}{} % add URL line breaks if available
\IfFileExists{bookmark.sty}{\usepackage{bookmark}}{\usepackage{hyperref}}
\hypersetup{
  pdftitle={AE 21: Multiple Regression, cont},
  pdfauthor={Dav King},
  hidelinks,
  pdfcreator={LaTeX via pandoc}}
\urlstyle{same} % disable monospaced font for URLs
\usepackage[margin=1in]{geometry}
\usepackage{color}
\usepackage{fancyvrb}
\newcommand{\VerbBar}{|}
\newcommand{\VERB}{\Verb[commandchars=\\\{\}]}
\DefineVerbatimEnvironment{Highlighting}{Verbatim}{commandchars=\\\{\}}
% Add ',fontsize=\small' for more characters per line
\usepackage{framed}
\definecolor{shadecolor}{RGB}{248,248,248}
\newenvironment{Shaded}{\begin{snugshade}}{\end{snugshade}}
\newcommand{\AlertTok}[1]{\textcolor[rgb]{0.94,0.16,0.16}{#1}}
\newcommand{\AnnotationTok}[1]{\textcolor[rgb]{0.56,0.35,0.01}{\textbf{\textit{#1}}}}
\newcommand{\AttributeTok}[1]{\textcolor[rgb]{0.77,0.63,0.00}{#1}}
\newcommand{\BaseNTok}[1]{\textcolor[rgb]{0.00,0.00,0.81}{#1}}
\newcommand{\BuiltInTok}[1]{#1}
\newcommand{\CharTok}[1]{\textcolor[rgb]{0.31,0.60,0.02}{#1}}
\newcommand{\CommentTok}[1]{\textcolor[rgb]{0.56,0.35,0.01}{\textit{#1}}}
\newcommand{\CommentVarTok}[1]{\textcolor[rgb]{0.56,0.35,0.01}{\textbf{\textit{#1}}}}
\newcommand{\ConstantTok}[1]{\textcolor[rgb]{0.00,0.00,0.00}{#1}}
\newcommand{\ControlFlowTok}[1]{\textcolor[rgb]{0.13,0.29,0.53}{\textbf{#1}}}
\newcommand{\DataTypeTok}[1]{\textcolor[rgb]{0.13,0.29,0.53}{#1}}
\newcommand{\DecValTok}[1]{\textcolor[rgb]{0.00,0.00,0.81}{#1}}
\newcommand{\DocumentationTok}[1]{\textcolor[rgb]{0.56,0.35,0.01}{\textbf{\textit{#1}}}}
\newcommand{\ErrorTok}[1]{\textcolor[rgb]{0.64,0.00,0.00}{\textbf{#1}}}
\newcommand{\ExtensionTok}[1]{#1}
\newcommand{\FloatTok}[1]{\textcolor[rgb]{0.00,0.00,0.81}{#1}}
\newcommand{\FunctionTok}[1]{\textcolor[rgb]{0.00,0.00,0.00}{#1}}
\newcommand{\ImportTok}[1]{#1}
\newcommand{\InformationTok}[1]{\textcolor[rgb]{0.56,0.35,0.01}{\textbf{\textit{#1}}}}
\newcommand{\KeywordTok}[1]{\textcolor[rgb]{0.13,0.29,0.53}{\textbf{#1}}}
\newcommand{\NormalTok}[1]{#1}
\newcommand{\OperatorTok}[1]{\textcolor[rgb]{0.81,0.36,0.00}{\textbf{#1}}}
\newcommand{\OtherTok}[1]{\textcolor[rgb]{0.56,0.35,0.01}{#1}}
\newcommand{\PreprocessorTok}[1]{\textcolor[rgb]{0.56,0.35,0.01}{\textit{#1}}}
\newcommand{\RegionMarkerTok}[1]{#1}
\newcommand{\SpecialCharTok}[1]{\textcolor[rgb]{0.00,0.00,0.00}{#1}}
\newcommand{\SpecialStringTok}[1]{\textcolor[rgb]{0.31,0.60,0.02}{#1}}
\newcommand{\StringTok}[1]{\textcolor[rgb]{0.31,0.60,0.02}{#1}}
\newcommand{\VariableTok}[1]{\textcolor[rgb]{0.00,0.00,0.00}{#1}}
\newcommand{\VerbatimStringTok}[1]{\textcolor[rgb]{0.31,0.60,0.02}{#1}}
\newcommand{\WarningTok}[1]{\textcolor[rgb]{0.56,0.35,0.01}{\textbf{\textit{#1}}}}
\usepackage{graphicx}
\makeatletter
\def\maxwidth{\ifdim\Gin@nat@width>\linewidth\linewidth\else\Gin@nat@width\fi}
\def\maxheight{\ifdim\Gin@nat@height>\textheight\textheight\else\Gin@nat@height\fi}
\makeatother
% Scale images if necessary, so that they will not overflow the page
% margins by default, and it is still possible to overwrite the defaults
% using explicit options in \includegraphics[width, height, ...]{}
\setkeys{Gin}{width=\maxwidth,height=\maxheight,keepaspectratio}
% Set default figure placement to htbp
\makeatletter
\def\fps@figure{htbp}
\makeatother
\setlength{\emergencystretch}{3em} % prevent overfull lines
\providecommand{\tightlist}{%
  \setlength{\itemsep}{0pt}\setlength{\parskip}{0pt}}
\setcounter{secnumdepth}{-\maxdimen} % remove section numbering
\ifLuaTeX
  \usepackage{selnolig}  % disable illegal ligatures
\fi

\begin{document}
\maketitle

\hypertarget{coming-up}{%
\subsection{Coming up}\label{coming-up}}

\begin{itemize}
\tightlist
\item
  Peer Review Due Friday at 9 PM.
\item
  Exam 2 released on Friday morning, due next Tuesday at 11:59 PM.
\item
  There will not be lab on Monday next week or class on Tuesday.
\end{itemize}

\hypertarget{main-ideas}{%
\subsection{Main ideas}\label{main-ideas}}

\begin{itemize}
\tightlist
\item
  Review and expand upon concepts from our first three regression
  classes.
\item
  Learn how to carry out and interpret multiple linear regressions.
\item
  Learn how to assess the conditions for inference in regression.
\end{itemize}

\hypertarget{packages}{%
\subsection{Packages}\label{packages}}

We'll use the \texttt{tidyverse}, \texttt{broom},
\texttt{fivethirtyeight}, and \texttt{viridis} packages again, as well
as the \texttt{car} package when calculate variance inflation factors
(VIFs) to examine whether our models have multicollinearity.

\hypertarget{please-recall}{%
\subsection{Please recall}\label{please-recall}}

\begin{itemize}
\item
  Response Variable: Variable whose behavior or variation you are trying
  to understand, on the y-axis. Also called the dependent variable.
\item
  Explanatory Variable: Other variables that you want to use to explain
  the variation in the response, on the x-axis. Also called independent
  variables, predictors, or features.
\item
  Predicted value: Output of the model function

  \begin{itemize}
  \tightlist
  \item
    The model function gives the typical value of the response variable
    conditioning on the explanatory variables (what does this mean?)
  \end{itemize}
\item
  Residuals: Shows how far each case is from its predicted value

  \begin{itemize}
  \tightlist
  \item
    Residual = Observed value - Predicted value
  \item
    Tells how far above/below the model function each case is
  \end{itemize}
\end{itemize}

\hypertarget{the-linear-model-with-a-single-predictor}{%
\subsection{The linear model with a single
predictor}\label{the-linear-model-with-a-single-predictor}}

\begin{itemize}
\tightlist
\item
  We're interested in the \(\beta_0\) (population parameter for the
  intercept) and the \(\beta_1\) (population parameter for the slope) in
  the following model:
\end{itemize}

\[ \hat{y} = \beta_0 + \beta_1~x + \epsilon \]

\begin{itemize}
\item
  Unfortunately, we can't get these values
\item
  So we use sample statistics to estimate them:
\end{itemize}

\[ \hat{y} = b_0 + b_1~x \]

\hypertarget{least-squares-regression}{%
\subsection{Least squares regression}\label{least-squares-regression}}

The regression line minimizes the sum of squared residuals.

\begin{itemize}
\item
  \textbf{Residuals}: \(e_i = y_i - \hat{y}_i\),
\item
  The regression line minimizes \(\sum_{i = 1}^n e_i^2\).
\item
  Equivalently, minimizing \(\sum_{i = 1}^n [y_i - (b_0 + b_1~x_i)]^2\)
\end{itemize}

\hypertarget{data-a-closer-look}{%
\subsection{Data: A Closer Look}\label{data-a-closer-look}}

\begin{Shaded}
\begin{Highlighting}[]
\NormalTok{sports\_car\_prices }\OtherTok{\textless{}{-}} \FunctionTok{read.csv}\NormalTok{(}\StringTok{"sportscars.csv"}\NormalTok{)}
\end{Highlighting}
\end{Shaded}

The file \texttt{sportscars.csv} contains prices for Porsche and Jaguar
cars for sale on cars.com.

\texttt{car}: car make (Jaguar or Porsche)

\texttt{price}: price in USD

\texttt{age}: age of the car in years

\texttt{mileage}: previous miles driven

\hypertarget{the-linear-model-with-a-single-predictor-1}{%
\subsection{The linear model with a single
predictor}\label{the-linear-model-with-a-single-predictor-1}}

\begin{Shaded}
\begin{Highlighting}[]
\NormalTok{prices\_model }\OtherTok{\textless{}{-}} \FunctionTok{linear\_reg}\NormalTok{() }\SpecialCharTok{\%\textgreater{}\%}
  \FunctionTok{set\_engine}\NormalTok{(}\StringTok{"lm"}\NormalTok{) }\SpecialCharTok{\%\textgreater{}\%}
  \FunctionTok{fit}\NormalTok{(price }\SpecialCharTok{\textasciitilde{}}\NormalTok{ age, }\AttributeTok{data =}\NormalTok{ sports\_car\_prices)}
\FunctionTok{tidy}\NormalTok{(prices\_model)}
\end{Highlighting}
\end{Shaded}

\begin{verbatim}
## # A tibble: 2 x 5
##   term        estimate std.error statistic  p.value
##   <chr>          <dbl>     <dbl>     <dbl>    <dbl>
## 1 (Intercept)   53246.     3322.     16.0  5.70e-23
## 2 age           -2149.      466.     -4.62 2.22e- 5
\end{verbatim}

But is the age the only variable that predicts price?

\hypertarget{the-linear-model-with-multiple-predictors}{%
\subsection{The linear model with multiple
predictors}\label{the-linear-model-with-multiple-predictors}}

\[ \hat{y} = \beta_0 + \beta_1~x_1 + \beta_2~x_2 + \cdots + \beta_k~x_k +\epsilon \]

\begin{itemize}
\tightlist
\item
  Sample model that we use to estimate the population model:
\end{itemize}

\[ \hat{y} = b_0 + b_1~x_1 + b_2~x_2 + \cdots + b_k~x_k \]

Let's add a variable.

\hypertarget{price-vs.-age-and-type-of-car}{%
\subsection{Price vs.~age and type of
car}\label{price-vs.-age-and-type-of-car}}

Does the relationship between price and age depend on type of car?

\begin{verbatim}
## `geom_smooth()` using formula 'y ~ x'
\end{verbatim}

\includegraphics{ae-21_files/figure-latex/plot1-1.pdf}

\hypertarget{modeling-with-main-effects}{%
\subsection{Modeling with main
effects}\label{modeling-with-main-effects}}

\begin{Shaded}
\begin{Highlighting}[]
\NormalTok{m\_main }\OtherTok{\textless{}{-}} \FunctionTok{linear\_reg}\NormalTok{() }\SpecialCharTok{\%\textgreater{}\%}
  \FunctionTok{set\_engine}\NormalTok{(}\StringTok{"lm"}\NormalTok{) }\SpecialCharTok{\%\textgreater{}\%}
  \FunctionTok{fit}\NormalTok{(price }\SpecialCharTok{\textasciitilde{}}\NormalTok{ age }\SpecialCharTok{+}\NormalTok{ car, }\AttributeTok{data =}\NormalTok{ sports\_car\_prices)}
\NormalTok{m\_main }\SpecialCharTok{\%\textgreater{}\%}
  \FunctionTok{tidy}\NormalTok{() }\SpecialCharTok{\%\textgreater{}\%}
  \FunctionTok{select}\NormalTok{(term, estimate)}
\end{Highlighting}
\end{Shaded}

\begin{verbatim}
## # A tibble: 3 x 2
##   term        estimate
##   <chr>          <dbl>
## 1 (Intercept)   44310.
## 2 age           -2487.
## 3 carPorsche    21648.
\end{verbatim}

Linear model:

\[ \widehat{price} = 44310 - 2487~age + 21648~carPorsche \]

\begin{itemize}
\item
  Plug in 0 for \texttt{carPorsche} to get the linear model for Jaguars.
\item
  Plug in 1 for \texttt{carPorsche} to get the linear model for
  Porsches.
\item
  Jaguar:
\end{itemize}

\[ \widehat{price} = 44310 - 2487~age + 21648* 0 \\
=44310-2487*age\]

\begin{itemize}
\item
  Porsche: \[ \widehat{price} = 44310 - 2487~age + 21648* 1 \\
  =65958 - 2487~age\]
\item
  Rate of change in price as the age of the car increases does not
  depend on make of car (same slopes)
\item
  Porsches are consistently more expensive than Jaguars (different
  intercepts)
\end{itemize}

\hypertarget{interpretation-of-main-effects}{%
\subsection{Interpretation of main
effects}\label{interpretation-of-main-effects}}

\includegraphics{ae-21_files/figure-latex/plot2-1.pdf}

\hypertarget{main-effects-numerical-and-categorical-predictors}{%
\subsection{Main effects, numerical and categorical
predictors}\label{main-effects-numerical-and-categorical-predictors}}

\begin{Shaded}
\begin{Highlighting}[]
\NormalTok{m\_main }\SpecialCharTok{\%\textgreater{}\%}
  \FunctionTok{tidy}\NormalTok{() }\SpecialCharTok{\%\textgreater{}\%}
  \FunctionTok{select}\NormalTok{(term, estimate)}
\end{Highlighting}
\end{Shaded}

\begin{verbatim}
## # A tibble: 3 x 2
##   term        estimate
##   <chr>          <dbl>
## 1 (Intercept)   44310.
## 2 age           -2487.
## 3 carPorsche    21648.
\end{verbatim}

\begin{Shaded}
\begin{Highlighting}[]
\NormalTok{m\_main\_coefs }\OtherTok{\textless{}{-}}\NormalTok{ m\_main }\SpecialCharTok{\%\textgreater{}\%}
  \FunctionTok{tidy}\NormalTok{() }\SpecialCharTok{\%\textgreater{}\%}
  \FunctionTok{select}\NormalTok{(term, estimate)}
\NormalTok{m\_main\_coefs}
\end{Highlighting}
\end{Shaded}

\begin{verbatim}
## # A tibble: 3 x 2
##   term        estimate
##   <chr>          <dbl>
## 1 (Intercept)   44310.
## 2 age           -2487.
## 3 carPorsche    21648.
\end{verbatim}

\begin{itemize}
\item
  \textbf{All else held constant}, for each additional year of a car's
  age, the price of the car is predicted to decrease, on average, by
  \$2,487.
\item
  \textbf{All else held constant}, Porsches are predicted, on average,
  to have a price that is \$21,648 greater than Jaguars.
\item
  Jaguars that have an age of 0 are predicted, on average, to have a
  price of \$44,310.
\end{itemize}

\hypertarget{what-went-wrong}{%
\subsection{What went wrong?}\label{what-went-wrong}}

\textbf{Question}: Why is our linear regression model different from
what we got from \texttt{geom\_smooth(method\ =\ "lm")}?

\begin{itemize}
\item
  \texttt{car} is the only variable in our model that affects the
  intercept.
\item
  The model we specified assumes Jaguars and Porsches have the
  \textbf{same slope} and \textbf{different intercepts}.
\item
  What is the most appropriate model for these data?

  \begin{itemize}
  \tightlist
  \item
    same slope and intercept for Jaguars and Porsches?
  \item
    same slope and different intercept for Jaguars and Porsches?
  \item
    different slope and different intercept for Jaguars and Porsches?
  \end{itemize}
\end{itemize}

Different slope and different intercepts is the most appropriate model.

\hypertarget{interacting-explanatory-variables}{%
\subsection{Interacting explanatory
variables}\label{interacting-explanatory-variables}}

\begin{itemize}
\item
  Including an interaction effect in the model allows for different
  slopes, i.e.~ nonparallel lines.
\item
  This means that the relationship between an explanatory variable and
  the response depends on another explanatory variable.
\item
  We can accomplish this by adding an \textbf{interaction variable}.
  This is the product of two explanatory variables (also sometimes
  called an interaction term).
\end{itemize}

\hypertarget{modeling-with-interaction-effects}{%
\subsection{Modeling with interaction
effects}\label{modeling-with-interaction-effects}}

\begin{Shaded}
\begin{Highlighting}[]
\FunctionTok{ggplot}\NormalTok{(}\AttributeTok{data =}\NormalTok{ sports\_car\_prices, }
       \FunctionTok{aes}\NormalTok{(}\AttributeTok{x =}\NormalTok{ age, }\AttributeTok{y =}\NormalTok{ price, }\AttributeTok{color =}\NormalTok{ car)) }\SpecialCharTok{+} 
    \FunctionTok{scale\_color\_viridis}\NormalTok{(}\AttributeTok{discrete =} \ConstantTok{TRUE}\NormalTok{, }\AttributeTok{option =} \StringTok{"D"}\NormalTok{, }\AttributeTok{name =} \StringTok{"Type of Car"}\NormalTok{) }\SpecialCharTok{+} 
  \FunctionTok{labs}\NormalTok{(}\AttributeTok{x =} \StringTok{"Age (years)"}\NormalTok{, }\AttributeTok{y =} \StringTok{"Price (USD)"}\NormalTok{, }\AttributeTok{color =} \StringTok{"Car Make"}\NormalTok{) }\SpecialCharTok{+} 
  \FunctionTok{geom\_point}\NormalTok{() }\SpecialCharTok{+} 
  \FunctionTok{geom\_smooth}\NormalTok{(}\AttributeTok{method=}\StringTok{"lm"}\NormalTok{, }\AttributeTok{se =} \ConstantTok{FALSE}\NormalTok{) }
\end{Highlighting}
\end{Shaded}

\begin{verbatim}
## `geom_smooth()` using formula 'y ~ x'
\end{verbatim}

\includegraphics{ae-21_files/figure-latex/interactingplot-1.pdf}

\begin{Shaded}
\begin{Highlighting}[]
\NormalTok{ m\_int }\OtherTok{\textless{}{-}} \FunctionTok{linear\_reg}\NormalTok{() }\SpecialCharTok{\%\textgreater{}\%}
  \FunctionTok{set\_engine}\NormalTok{(}\StringTok{"lm"}\NormalTok{) }\SpecialCharTok{\%\textgreater{}\%}
  \FunctionTok{fit}\NormalTok{(price }\SpecialCharTok{\textasciitilde{}}\NormalTok{ age }\SpecialCharTok{+}\NormalTok{ car }\SpecialCharTok{+}\NormalTok{ age }\SpecialCharTok{*}\NormalTok{ car, }
            \AttributeTok{data =}\NormalTok{ sports\_car\_prices) }
\NormalTok{m\_int }\SpecialCharTok{\%\textgreater{}\%}
  \FunctionTok{tidy}\NormalTok{() }\SpecialCharTok{\%\textgreater{}\%}
  \FunctionTok{select}\NormalTok{(term, estimate)}
\end{Highlighting}
\end{Shaded}

\begin{verbatim}
## # A tibble: 4 x 2
##   term           estimate
##   <chr>             <dbl>
## 1 (Intercept)      56988.
## 2 age              -5040.
## 3 carPorsche        6387.
## 4 age:carPorsche    2969.
\end{verbatim}

\[\widehat{price} = 56988 - 5040~age + 6387~carPorsche + 2969~age \times carPorsche\]

\hypertarget{interpretation-of-interaction-effects}{%
\subsection{Interpretation of interaction
effects}\label{interpretation-of-interaction-effects}}

\[\widehat{price} = 56988 - 5040~age + 6387~carPorsche + 2969~age \times carPorsche\]

\begin{itemize}
\item
  Plug in 0 for \texttt{carPorsche} to get the linear model for Jaguars.
\item
  Plug in 1 for \texttt{carPorsche} to get the linear model for
  Porsches.
\item
  Jaguar:
\end{itemize}

\[\begin{align}\widehat{price} &= 56988 - 5040~age + 6387~carPorsche + 2969~age \times carPorsche \\
&= 56988 - 5040~age + 6387 \times 0 + 2969~age \times 0\\
&= 56988 - 5040~age\end{align}\]

\begin{itemize}
\tightlist
\item
  Porsche:
\end{itemize}

\[\begin{align}\widehat{price} &= 56988 - 5040~age + 6387~carPorsche + 2969~age \times carPorsche \\
&= 56988 - 5040~age + 6387 \times 1 + 2969~age \times 1\\
&= 63375 - 2071~age\end{align}\]

\hypertarget{interpretation-of-interaction-effects-1}{%
\subsection{Interpretation of interaction
effects}\label{interpretation-of-interaction-effects-1}}

\begin{itemize}
\tightlist
\item
  Jaguar:
\end{itemize}

\[\widehat{price} = 56988 - 5040~age\]

\begin{itemize}
\tightlist
\item
  Porsche:
\end{itemize}

\[\widehat{price} = 63375 - 2071~age\]

\begin{itemize}
\item
  Rate of change in price as the age of the car increases depends on the
  make of the car (different slopes).
\item
  Porsches are consistently more expensive than Jaguars (different
  intercepts).
\end{itemize}

\hypertarget{continuous-by-continuous-interactions}{%
\subsection{Continuous by continuous
interactions}\label{continuous-by-continuous-interactions}}

\begin{itemize}
\item
  Interpretation becomes trickier
\item
  Slopes conditional on values of explanatory variables
\end{itemize}

\hypertarget{third-order-interactions}{%
\subsection{Third order interactions}\label{third-order-interactions}}

\begin{itemize}
\item
  Can you? Yes
\item
  Should you? Probably not if you want to interpret these interactions
  in context of the data.
\end{itemize}

\hypertarget{multicollinearity}{%
\subsection{Multicollinearity}\label{multicollinearity}}

You don't want the predictors to be too correlated with each other in a
multiple regression model. When they are correlated with each other, you
have \textbf{mutlicollinearity}. One way to diagnose multicollinearity
is with \textbf{variance inflation factors.} There's no specific cutoff,
but a VIF of 10 if sometimes used as a cutoff.

Let's see if we have multicollinearity in our first model. (Please do
not run linear regressions this way on the exam, I'm doing it this way
to demonstrate with \texttt{car}.)

\begin{Shaded}
\begin{Highlighting}[]
\NormalTok{m\_main\_2 }\OtherTok{\textless{}{-}} \FunctionTok{lm}\NormalTok{(price }\SpecialCharTok{\textasciitilde{}}\NormalTok{ age }\SpecialCharTok{+}\NormalTok{ car, }\AttributeTok{data =}\NormalTok{ sports\_car\_prices)}
\FunctionTok{tibble}\NormalTok{(}\FunctionTok{vif}\NormalTok{(m\_main\_2))}
\end{Highlighting}
\end{Shaded}

\begin{verbatim}
## # A tibble: 2 x 1
##   `vif(m_main_2)`
##             <dbl>
## 1            1.02
## 2            1.02
\end{verbatim}

Now, let's check if for the interactive model.

\begin{Shaded}
\begin{Highlighting}[]
\NormalTok{m\_int\_2 }\OtherTok{\textless{}{-}} \FunctionTok{lm}\NormalTok{(price }\SpecialCharTok{\textasciitilde{}}\NormalTok{ age }\SpecialCharTok{+}\NormalTok{ car }\SpecialCharTok{+}\NormalTok{ age }\SpecialCharTok{*}\NormalTok{ car, }
            \AttributeTok{data =}\NormalTok{ sports\_car\_prices)}
\FunctionTok{tibble}\NormalTok{(}\FunctionTok{vif}\NormalTok{(m\_int\_2))}
\end{Highlighting}
\end{Shaded}

\begin{verbatim}
## # A tibble: 3 x 1
##   `vif(m_int_2)`
##            <dbl>
## 1           7.27
## 2           3.85
## 3          11.2
\end{verbatim}

Notice the VIFs here are higher. This is to be expected with an
interactive model.

\textbf{Question}: Why do you think VIFs will be higher in interactive
models?

Yes- this is because the interaction terms are made up of the other
variables. So, they are likely to correlate with them. This is normal.

\hypertarget{p-hacking}{%
\subsection{P-Hacking}\label{p-hacking}}

Let's return to the candy rankings data:

Notice below that our p-value for the sugar percentile variable is
\textbf{REALLY CLOSE} to 0.05. While arbitrary, this threshold is
important in statistics and is sometimes used as the cutoff to determine
whether results are publishable.

\begin{Shaded}
\begin{Highlighting}[]
\NormalTok{candy\_rankings }\OtherTok{\textless{}{-}}\NormalTok{ candy\_rankings }\SpecialCharTok{\%\textgreater{}\%}
  \FunctionTok{mutate}\NormalTok{(}\AttributeTok{sugarpercent100 =}\NormalTok{ sugarpercent }\SpecialCharTok{*} \DecValTok{100}\NormalTok{)}
\NormalTok{candy }\OtherTok{\textless{}{-}} \FunctionTok{linear\_reg}\NormalTok{() }\SpecialCharTok{\%\textgreater{}\%}
  \FunctionTok{set\_engine}\NormalTok{(}\StringTok{"lm"}\NormalTok{) }\SpecialCharTok{\%\textgreater{}\%}
  \FunctionTok{fit}\NormalTok{(winpercent }\SpecialCharTok{\textasciitilde{}}\NormalTok{ sugarpercent100 }\SpecialCharTok{+}\NormalTok{ chocolate, }\AttributeTok{data =}\NormalTok{ candy\_rankings) }
\FunctionTok{tidy}\NormalTok{(candy)}
\end{Highlighting}
\end{Shaded}

\begin{verbatim}
## # A tibble: 3 x 5
##   term            estimate std.error statistic  p.value
##   <chr>              <dbl>     <dbl>     <dbl>    <dbl>
## 1 (Intercept)      38.3       2.55       15.0  3.31e-25
## 2 sugarpercent100   0.0857    0.0435      1.97 5.25e- 2
## 3 chocolateTRUE    18.3       2.47        7.40 1.06e-10
\end{verbatim}

\href{https://projects.fivethirtyeight.com/p-hacking}{FiveThirtyEight}
has a interactive page showing how you can p-hack your way to
publication.

How could we p-hack our way to a significant result for the sugar
variable?

Try adding the variable for whether the variable is a hard candy.

\begin{Shaded}
\begin{Highlighting}[]
\NormalTok{hard\_candy }\OtherTok{\textless{}{-}} \FunctionTok{linear\_reg}\NormalTok{() }\SpecialCharTok{\%\textgreater{}\%} 
  \FunctionTok{set\_engine}\NormalTok{(}\StringTok{"lm"}\NormalTok{) }\SpecialCharTok{\%\textgreater{}\%} 
  \FunctionTok{fit}\NormalTok{(winpercent }\SpecialCharTok{\textasciitilde{}}\NormalTok{ sugarpercent100 }\SpecialCharTok{+}\NormalTok{ chocolate }\SpecialCharTok{+}\NormalTok{ hard, }\AttributeTok{data =}\NormalTok{ candy\_rankings)}
\FunctionTok{tidy}\NormalTok{(hard\_candy)}
\end{Highlighting}
\end{Shaded}

\begin{verbatim}
## # A tibble: 4 x 5
##   term            estimate std.error statistic  p.value
##   <chr>              <dbl>     <dbl>     <dbl>    <dbl>
## 1 (Intercept)      39.3       2.64       14.9  6.05e-25
## 2 sugarpercent100   0.0944    0.0437      2.16 3.36e- 2
## 3 chocolateTRUE    16.9       2.63        6.44 7.98e- 9
## 4 hardTRUE         -4.98      3.41       -1.46 1.48e- 1
\end{verbatim}

We now have statistically significant results for our main variable of
interest! But there are some data ethics issues with this approach.

\textbf{Question}: Why is p-hacking problematic?

This can result in you getting statistically significant results when
you may not actually have them, increasing the probability of a Type 1
Error, among other concerns.

\hypertarget{practice}{%
\subsection{Practice}\label{practice}}

\begin{enumerate}
\def\labelenumi{\arabic{enumi}.}
\tightlist
\item
  Run and interpret a multiple regression with both age and mileage as
  predictors. Are both of these statistically significant predictors of
  the price of a car?
\end{enumerate}

\begin{Shaded}
\begin{Highlighting}[]
\NormalTok{age\_mile }\OtherTok{\textless{}{-}} \FunctionTok{linear\_reg}\NormalTok{() }\SpecialCharTok{\%\textgreater{}\%} 
  \FunctionTok{set\_engine}\NormalTok{(}\StringTok{"lm"}\NormalTok{) }\SpecialCharTok{\%\textgreater{}\%} 
  \FunctionTok{fit}\NormalTok{(price }\SpecialCharTok{\textasciitilde{}}\NormalTok{ age }\SpecialCharTok{+}\NormalTok{ mileage, }\AttributeTok{data =}\NormalTok{ sports\_car\_prices)}
\CommentTok{\#note that the instructions did not say to examine interaction effects, but }
\CommentTok{\#see below.}
\FunctionTok{tidy}\NormalTok{(age\_mile)}
\end{Highlighting}
\end{Shaded}

\begin{verbatim}
## # A tibble: 3 x 5
##   term         estimate std.error statistic  p.value
##   <chr>           <dbl>     <dbl>     <dbl>    <dbl>
## 1 (Intercept) 62950.     3176.       19.8   3.03e-27
## 2 age           516.      601.        0.859 3.94e- 1
## 3 mileage        -0.695     0.122    -5.68  4.75e- 7
\end{verbatim}

\begin{Shaded}
\begin{Highlighting}[]
\NormalTok{age\_mile\_interact }\OtherTok{\textless{}{-}} \FunctionTok{linear\_reg}\NormalTok{() }\SpecialCharTok{\%\textgreater{}\%} 
  \FunctionTok{set\_engine}\NormalTok{(}\StringTok{"lm"}\NormalTok{) }\SpecialCharTok{\%\textgreater{}\%} 
  \FunctionTok{fit}\NormalTok{(price }\SpecialCharTok{\textasciitilde{}}\NormalTok{ age }\SpecialCharTok{*}\NormalTok{ mileage, }\AttributeTok{data =}\NormalTok{ sports\_car\_prices)}
\FunctionTok{tidy}\NormalTok{(age\_mile\_interact)}
\end{Highlighting}
\end{Shaded}

\begin{verbatim}
## # A tibble: 4 x 5
##   term          estimate std.error statistic  p.value
##   <chr>            <dbl>     <dbl>     <dbl>    <dbl>
## 1 (Intercept) 70809.     4441.         15.9  1.78e-22
## 2 age         -1807.     1115.         -1.62 1.11e- 1
## 3 mileage        -0.808     0.126      -6.40 3.35e- 8
## 4 age:mileage     0.0334    0.0137      2.43 1.82e- 2
\end{verbatim}

Slopes: For main interaction effects: for each year of age, we predict a
\$516 increase in price. For each additional mile on the car, we predict
a decrease of \$0.695 in price.

Intercept: We expect a car with zero years of age and zero miles to cost
\$62950.

\begin{enumerate}
\def\labelenumi{\arabic{enumi}.}
\setcounter{enumi}{1}
\tightlist
\item
  (To Review) Find and interpret the adjusted \(R^2\) for this model.
\end{enumerate}

\begin{Shaded}
\begin{Highlighting}[]
\FunctionTok{glance}\NormalTok{(age\_mile)}\SpecialCharTok{$}\NormalTok{adj.r.squared}
\end{Highlighting}
\end{Shaded}

\begin{verbatim}
## [1] 0.5167014
\end{verbatim}

\begin{Shaded}
\begin{Highlighting}[]
\FunctionTok{glance}\NormalTok{(age\_mile\_interact)}\SpecialCharTok{$}\NormalTok{adj.r.squared}
\end{Highlighting}
\end{Shaded}

\begin{verbatim}
## [1] 0.5550869
\end{verbatim}

51.7\% of the variance in car price can be explained by its age and
mileage, and 55.5\% of that same variance can be explained by the
interaction of its age and mileage. This also suggests that including
the interaction effects makes for a more effective model of car price.

\begin{enumerate}
\def\labelenumi{\arabic{enumi}.}
\setcounter{enumi}{2}
\tightlist
\item
  Examine the extent to which there is multicollinearity in this model.
\end{enumerate}

\begin{Shaded}
\begin{Highlighting}[]
\NormalTok{age\_mile\_lm }\OtherTok{\textless{}{-}} \FunctionTok{lm}\NormalTok{(price }\SpecialCharTok{\textasciitilde{}}\NormalTok{ age }\SpecialCharTok{+}\NormalTok{ mileage, }\AttributeTok{data =}\NormalTok{ sports\_car\_prices)}
\NormalTok{age\_mile\_interact\_lm }\OtherTok{\textless{}{-}} \FunctionTok{lm}\NormalTok{(price }\SpecialCharTok{\textasciitilde{}}\NormalTok{ age }\SpecialCharTok{+}\NormalTok{ mileage }\SpecialCharTok{+}\NormalTok{ age }\SpecialCharTok{*}\NormalTok{ mileage,}
                           \AttributeTok{data =}\NormalTok{ sports\_car\_prices)}
\FunctionTok{vif}\NormalTok{(age\_mile\_lm)}
\end{Highlighting}
\end{Shaded}

\begin{verbatim}
##      age  mileage 
## 2.562603 2.562603
\end{verbatim}

\begin{Shaded}
\begin{Highlighting}[]
\FunctionTok{vif}\NormalTok{(age\_mile\_interact\_lm)}
\end{Highlighting}
\end{Shaded}

\begin{verbatim}
##         age     mileage age:mileage 
##    9.594751    2.967861   11.073879
\end{verbatim}

There is a fair bit of multicollinearity in this model, which is to be
expected.

Now, please turn to the dataset in \texttt{nycairquality.csv}. This file
contains daily air quality measurements in New York from May to
September 1973 and collected by the New York State Department of
Conservation and the National Weather Service (Chambers, J. M.,
Cleveland, W. S., Kleiner, B. and Tukey, P. A. (1983) \emph{Graphical
Methods for Data Analysis}. Belmont, CA: Wadsworth).

\begin{itemize}
\tightlist
\item
  \texttt{Ozone}: ozone (ppb)
\item
  \texttt{Solar.R}: solar radiation (langleys)
\item
  \texttt{Wind}: wind (mph)
\item
  \texttt{Temp}: temperature (degrees F)
\end{itemize}

\begin{Shaded}
\begin{Highlighting}[]
\NormalTok{airquality }\OtherTok{\textless{}{-}} \FunctionTok{read.csv}\NormalTok{(}\StringTok{"airquality.csv"}\NormalTok{)}
\end{Highlighting}
\end{Shaded}

\begin{enumerate}
\def\labelenumi{\arabic{enumi}.}
\setcounter{enumi}{3}
\tightlist
\item
  Please run and interpret a model with ozone in parts per billion as
  the response variable and solar radiation, wind, and temperature as
  the explanatory variables.
\end{enumerate}

\begin{Shaded}
\begin{Highlighting}[]
\NormalTok{ozone\_model }\OtherTok{\textless{}{-}} \FunctionTok{linear\_reg}\NormalTok{() }\SpecialCharTok{\%\textgreater{}\%} 
  \FunctionTok{set\_engine}\NormalTok{(}\StringTok{"lm"}\NormalTok{) }\SpecialCharTok{\%\textgreater{}\%} 
  \FunctionTok{fit}\NormalTok{(Ozone }\SpecialCharTok{\textasciitilde{}}\NormalTok{ Solar.R }\SpecialCharTok{+}\NormalTok{ Wind }\SpecialCharTok{+}\NormalTok{ Temp, }\AttributeTok{data =}\NormalTok{ airquality)}
\FunctionTok{tidy}\NormalTok{(ozone\_model)}
\end{Highlighting}
\end{Shaded}

\begin{verbatim}
## # A tibble: 4 x 5
##   term        estimate std.error statistic       p.value
##   <chr>          <dbl>     <dbl>     <dbl>         <dbl>
## 1 (Intercept) -64.3      23.1        -2.79 0.00623      
## 2 Solar.R       0.0598    0.0232      2.58 0.0112       
## 3 Wind         -3.33      0.654      -5.09 0.00000152   
## 4 Temp          1.65      0.254       6.52 0.00000000242
\end{verbatim}

Slopes: For each unit increase in solar radiation, we predict an
increase of 0.0598 parts per billion ozone. For each additional mile per
hour of wind, we predict a decrease of -3.33 parts per billion ozone.
For each Fahrenheit degree of temperature, we predict an increase of
1.65 parts per billion ozone.

Intercept: With zero solar radiation, zero wind, and at zero degrees
Fahrenheit, we predict -64.3 parts per billion ozone.

\end{document}
