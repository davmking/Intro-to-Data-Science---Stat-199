% Options for packages loaded elsewhere
\PassOptionsToPackage{unicode}{hyperref}
\PassOptionsToPackage{hyphens}{url}
%
\documentclass[
]{article}
\title{Flint Water Crisis}
\author{Dav King}
\date{1/11/2022}

\usepackage{amsmath,amssymb}
\usepackage{lmodern}
\usepackage{iftex}
\ifPDFTeX
  \usepackage[T1]{fontenc}
  \usepackage[utf8]{inputenc}
  \usepackage{textcomp} % provide euro and other symbols
\else % if luatex or xetex
  \usepackage{unicode-math}
  \defaultfontfeatures{Scale=MatchLowercase}
  \defaultfontfeatures[\rmfamily]{Ligatures=TeX,Scale=1}
\fi
% Use upquote if available, for straight quotes in verbatim environments
\IfFileExists{upquote.sty}{\usepackage{upquote}}{}
\IfFileExists{microtype.sty}{% use microtype if available
  \usepackage[]{microtype}
  \UseMicrotypeSet[protrusion]{basicmath} % disable protrusion for tt fonts
}{}
\makeatletter
\@ifundefined{KOMAClassName}{% if non-KOMA class
  \IfFileExists{parskip.sty}{%
    \usepackage{parskip}
  }{% else
    \setlength{\parindent}{0pt}
    \setlength{\parskip}{6pt plus 2pt minus 1pt}}
}{% if KOMA class
  \KOMAoptions{parskip=half}}
\makeatother
\usepackage{xcolor}
\IfFileExists{xurl.sty}{\usepackage{xurl}}{} % add URL line breaks if available
\IfFileExists{bookmark.sty}{\usepackage{bookmark}}{\usepackage{hyperref}}
\hypersetup{
  pdftitle={Flint Water Crisis},
  pdfauthor={Dav King},
  hidelinks,
  pdfcreator={LaTeX via pandoc}}
\urlstyle{same} % disable monospaced font for URLs
\usepackage[margin=1in]{geometry}
\usepackage{color}
\usepackage{fancyvrb}
\newcommand{\VerbBar}{|}
\newcommand{\VERB}{\Verb[commandchars=\\\{\}]}
\DefineVerbatimEnvironment{Highlighting}{Verbatim}{commandchars=\\\{\}}
% Add ',fontsize=\small' for more characters per line
\usepackage{framed}
\definecolor{shadecolor}{RGB}{248,248,248}
\newenvironment{Shaded}{\begin{snugshade}}{\end{snugshade}}
\newcommand{\AlertTok}[1]{\textcolor[rgb]{0.94,0.16,0.16}{#1}}
\newcommand{\AnnotationTok}[1]{\textcolor[rgb]{0.56,0.35,0.01}{\textbf{\textit{#1}}}}
\newcommand{\AttributeTok}[1]{\textcolor[rgb]{0.77,0.63,0.00}{#1}}
\newcommand{\BaseNTok}[1]{\textcolor[rgb]{0.00,0.00,0.81}{#1}}
\newcommand{\BuiltInTok}[1]{#1}
\newcommand{\CharTok}[1]{\textcolor[rgb]{0.31,0.60,0.02}{#1}}
\newcommand{\CommentTok}[1]{\textcolor[rgb]{0.56,0.35,0.01}{\textit{#1}}}
\newcommand{\CommentVarTok}[1]{\textcolor[rgb]{0.56,0.35,0.01}{\textbf{\textit{#1}}}}
\newcommand{\ConstantTok}[1]{\textcolor[rgb]{0.00,0.00,0.00}{#1}}
\newcommand{\ControlFlowTok}[1]{\textcolor[rgb]{0.13,0.29,0.53}{\textbf{#1}}}
\newcommand{\DataTypeTok}[1]{\textcolor[rgb]{0.13,0.29,0.53}{#1}}
\newcommand{\DecValTok}[1]{\textcolor[rgb]{0.00,0.00,0.81}{#1}}
\newcommand{\DocumentationTok}[1]{\textcolor[rgb]{0.56,0.35,0.01}{\textbf{\textit{#1}}}}
\newcommand{\ErrorTok}[1]{\textcolor[rgb]{0.64,0.00,0.00}{\textbf{#1}}}
\newcommand{\ExtensionTok}[1]{#1}
\newcommand{\FloatTok}[1]{\textcolor[rgb]{0.00,0.00,0.81}{#1}}
\newcommand{\FunctionTok}[1]{\textcolor[rgb]{0.00,0.00,0.00}{#1}}
\newcommand{\ImportTok}[1]{#1}
\newcommand{\InformationTok}[1]{\textcolor[rgb]{0.56,0.35,0.01}{\textbf{\textit{#1}}}}
\newcommand{\KeywordTok}[1]{\textcolor[rgb]{0.13,0.29,0.53}{\textbf{#1}}}
\newcommand{\NormalTok}[1]{#1}
\newcommand{\OperatorTok}[1]{\textcolor[rgb]{0.81,0.36,0.00}{\textbf{#1}}}
\newcommand{\OtherTok}[1]{\textcolor[rgb]{0.56,0.35,0.01}{#1}}
\newcommand{\PreprocessorTok}[1]{\textcolor[rgb]{0.56,0.35,0.01}{\textit{#1}}}
\newcommand{\RegionMarkerTok}[1]{#1}
\newcommand{\SpecialCharTok}[1]{\textcolor[rgb]{0.00,0.00,0.00}{#1}}
\newcommand{\SpecialStringTok}[1]{\textcolor[rgb]{0.31,0.60,0.02}{#1}}
\newcommand{\StringTok}[1]{\textcolor[rgb]{0.31,0.60,0.02}{#1}}
\newcommand{\VariableTok}[1]{\textcolor[rgb]{0.00,0.00,0.00}{#1}}
\newcommand{\VerbatimStringTok}[1]{\textcolor[rgb]{0.31,0.60,0.02}{#1}}
\newcommand{\WarningTok}[1]{\textcolor[rgb]{0.56,0.35,0.01}{\textbf{\textit{#1}}}}
\usepackage{graphicx}
\makeatletter
\def\maxwidth{\ifdim\Gin@nat@width>\linewidth\linewidth\else\Gin@nat@width\fi}
\def\maxheight{\ifdim\Gin@nat@height>\textheight\textheight\else\Gin@nat@height\fi}
\makeatother
% Scale images if necessary, so that they will not overflow the page
% margins by default, and it is still possible to overwrite the defaults
% using explicit options in \includegraphics[width, height, ...]{}
\setkeys{Gin}{width=\maxwidth,height=\maxheight,keepaspectratio}
% Set default figure placement to htbp
\makeatletter
\def\fps@figure{htbp}
\makeatother
\setlength{\emergencystretch}{3em} % prevent overfull lines
\providecommand{\tightlist}{%
  \setlength{\itemsep}{0pt}\setlength{\parskip}{0pt}}
\setcounter{secnumdepth}{-\maxdimen} % remove section numbering
\ifLuaTeX
  \usepackage{selnolig}  % disable illegal ligatures
\fi

\begin{document}
\maketitle

\hypertarget{introduction}{%
\subsection{Introduction}\label{introduction}}

The data set consists of 271 homes sampled with three water lead
contaminant values at designated time points. The lead content is in
parts per billion (ppb). Additionally, some location data is given about
each home.

\hypertarget{packages}{%
\subsection{Packages}\label{packages}}

\begin{Shaded}
\begin{Highlighting}[]
\FunctionTok{library}\NormalTok{(tidyverse)}
\end{Highlighting}
\end{Shaded}

\hypertarget{data}{%
\subsection{Data}\label{data}}

To get started, read in the \texttt{flint.csv} file using the function
\texttt{read\_csv}. First, use the Upload button under Files.

\begin{Shaded}
\begin{Highlighting}[]
\NormalTok{flint }\OtherTok{\textless{}{-}} \FunctionTok{read\_csv}\NormalTok{(}\StringTok{"flint.csv"}\NormalTok{)}
\end{Highlighting}
\end{Shaded}

In this file, there are five variables:

\begin{itemize}
\tightlist
\item
  \textbf{id}: sample ID number
\item
  \textbf{zip}: ZIP code in Flint of the sample's location
\item
  \textbf{ward}: ward in Flint of the sample's location
\item
  \textbf{draw}: which time point the water was sampled from
\item
  \textbf{lead}: lead content in parts per billion
\end{itemize}

Let's preview the data with the \texttt{glimpse()} function:

\begin{Shaded}
\begin{Highlighting}[]
\FunctionTok{glimpse}\NormalTok{(flint)}
\end{Highlighting}
\end{Shaded}

\begin{verbatim}
## Rows: 813
## Columns: 5
## $ id   <dbl> 1, 2, 4, 5, 6, 7, 8, 9, 12, 13, 15, 16, 17, 18, 19, 20, 21, 22, 2~
## $ zip  <dbl> 48504, 48507, 48504, 48507, 48505, 48507, 48507, 48503, 48507, 48~
## $ ward <dbl> 6, 9, 1, 8, 3, 9, 9, 5, 9, 3, 9, 5, 2, 7, 9, 9, 5, 6, 2, 6, 1, 5,~
## $ lead <dbl> 0.344, 8.133, 1.111, 8.007, 1.951, 7.200, 40.630, 1.100, 10.600, ~
## $ draw <chr> "first", "first", "first", "first", "first", "first", "first", "f~
\end{verbatim}

\hypertarget{analysis}{%
\subsection{Analysis}\label{analysis}}

\hypertarget{part-1}{%
\subsubsection{Part 1}\label{part-1}}

Let's see how many samples were taken from each zip code.

\begin{Shaded}
\begin{Highlighting}[]
\NormalTok{flint }\SpecialCharTok{\%\textgreater{}\%}               \CommentTok{\# data}
  \FunctionTok{group\_by}\NormalTok{(zip) }\SpecialCharTok{\%\textgreater{}\%}     \CommentTok{\# perform a grouping by zip code}
  \FunctionTok{count}\NormalTok{()               }\CommentTok{\# count occurrences}
\end{Highlighting}
\end{Shaded}

\begin{verbatim}
## # A tibble: 8 x 2
## # Groups:   zip [8]
##     zip     n
##   <dbl> <int>
## 1 48502     3
## 2 48503   207
## 3 48504   165
## 4 48505   144
## 5 48506   132
## 6 48507   153
## 7 48529     3
## 8 48532     6
\end{verbatim}

48503 had the most samples drawn, at 207.

\hypertarget{part-2}{%
\subsubsection{Part 2}\label{part-2}}

Next, let's look at the mean and median lead contaminant values for each
zip code and draw combination. We have eight zip codes and samples taken
at three times. How many combinations do we have?

\begin{Shaded}
\begin{Highlighting}[]
\NormalTok{flint }\SpecialCharTok{\%\textgreater{}\%} 
  \FunctionTok{group\_by}\NormalTok{(zip, draw) }\SpecialCharTok{\%\textgreater{}\%} 
  \FunctionTok{summarise}\NormalTok{(}\AttributeTok{mean\_pb =} \FunctionTok{mean}\NormalTok{(lead))}
\end{Highlighting}
\end{Shaded}

\begin{verbatim}
## `summarise()` has grouped output by 'zip'. You can override using the `.groups` argument.
\end{verbatim}

\begin{verbatim}
## # A tibble: 24 x 3
## # Groups:   zip [8]
##      zip draw   mean_pb
##    <dbl> <chr>    <dbl>
##  1 48502 first     2.27
##  2 48502 second    2.81
##  3 48502 third     3.05
##  4 48503 first    11.0 
##  5 48503 second    5.66
##  6 48503 third     3.77
##  7 48504 first    13.2 
##  8 48504 second   32.6 
##  9 48504 third     5.13
## 10 48505 first     6.09
## # ... with 14 more rows
\end{verbatim}

\begin{Shaded}
\begin{Highlighting}[]
\NormalTok{flint }\SpecialCharTok{\%\textgreater{}\%} 
  \FunctionTok{group\_by}\NormalTok{(zip, draw) }\SpecialCharTok{\%\textgreater{}\%} 
  \FunctionTok{summarise}\NormalTok{(}\AttributeTok{median\_pb =} \FunctionTok{median}\NormalTok{(lead))}
\end{Highlighting}
\end{Shaded}

\begin{verbatim}
## `summarise()` has grouped output by 'zip'. You can override using the `.groups` argument.
\end{verbatim}

\begin{verbatim}
## # A tibble: 24 x 3
## # Groups:   zip [8]
##      zip draw   median_pb
##    <dbl> <chr>      <dbl>
##  1 48502 first      2.27 
##  2 48502 second     2.81 
##  3 48502 third      3.05 
##  4 48503 first      5.15 
##  5 48503 second     2.47 
##  6 48503 third      1.23 
##  7 48504 first      2.83 
##  8 48504 second     1.22 
##  9 48504 third      0.744
## 10 48505 first      3.32 
## # ... with 14 more rows
\end{verbatim}

How many rows are in each of two above data frames? Each one has 24
rows. This is from 8 * 3.

\hypertarget{part-3}{%
\subsubsection{Part 3}\label{part-3}}

Modify the code below to compute the mean and median lead contaminant
values for zip code 48503 at the first draw. What should you put in for
\texttt{draw\ ==\ "\_\_\_\_\_"}? Don't forget to uncomment the second
line of code.

\begin{Shaded}
\begin{Highlighting}[]
\NormalTok{flint }\SpecialCharTok{\%\textgreater{}\%} 
  \FunctionTok{filter}\NormalTok{(zip }\SpecialCharTok{==} \DecValTok{48503}\NormalTok{, draw }\SpecialCharTok{==} \StringTok{"first"}\NormalTok{) }\SpecialCharTok{\%\textgreater{}\%} 
  \FunctionTok{summarise}\NormalTok{(}\AttributeTok{mean\_pb =} \FunctionTok{mean}\NormalTok{(lead),}
            \AttributeTok{median\_pb =} \FunctionTok{median}\NormalTok{(lead))}
\end{Highlighting}
\end{Shaded}

\begin{verbatim}
## # A tibble: 1 x 2
##   mean_pb median_pb
##     <dbl>     <dbl>
## 1    11.0      5.15
\end{verbatim}

\hypertarget{part-4}{%
\subsubsection{Part 4}\label{part-4}}

Let's make some plots, where we will focus on zip codes 48503, 48504,
48505, 48506, and 48507. We will restrict our attention to samples with
lead values less than 1,000 ppb.

\begin{Shaded}
\begin{Highlighting}[]
\NormalTok{flint\_focus }\OtherTok{\textless{}{-}}\NormalTok{ flint }\SpecialCharTok{\%\textgreater{}\%} 
  \FunctionTok{filter}\NormalTok{(zip }\SpecialCharTok{\%in\%} \DecValTok{48503}\SpecialCharTok{:}\DecValTok{48507}\NormalTok{, lead }\SpecialCharTok{\textless{}} \DecValTok{1000}\NormalTok{)}
\end{Highlighting}
\end{Shaded}

Below are side-by-side box plots for the three flushing times in each of
the five zip codes we considered. Add \texttt{x} and \texttt{y} labels;
add a title by inserting \texttt{title\ =\ "title\_name"} inside the
\texttt{labs()} function.

\begin{Shaded}
\begin{Highlighting}[]
\FunctionTok{ggplot}\NormalTok{(}\AttributeTok{data =}\NormalTok{ flint\_focus, }\FunctionTok{aes}\NormalTok{(}\AttributeTok{x =} \FunctionTok{factor}\NormalTok{(zip), }\AttributeTok{y =}\NormalTok{ lead)) }\SpecialCharTok{+}
  \FunctionTok{geom\_boxplot}\NormalTok{(}\FunctionTok{aes}\NormalTok{(}\AttributeTok{fill =} \FunctionTok{factor}\NormalTok{(draw))) }\SpecialCharTok{+}
  \FunctionTok{labs}\NormalTok{(}\AttributeTok{x =} \StringTok{"Zip Code"}\NormalTok{, }\AttributeTok{y =} \StringTok{"Lead (ppb)"}\NormalTok{, }\AttributeTok{title =} \StringTok{"Flint, MI Lead Content by Zip Code"}\NormalTok{, }\AttributeTok{fill =} \StringTok{"Flushing time"}\NormalTok{) }\SpecialCharTok{+}
  \FunctionTok{scale\_fill\_discrete}\NormalTok{(}\AttributeTok{breaks =} \FunctionTok{c}\NormalTok{(}\StringTok{"first"}\NormalTok{, }\StringTok{"second"}\NormalTok{, }\StringTok{"third"}\NormalTok{),}
                      \AttributeTok{labels =} \FunctionTok{c}\NormalTok{(}\StringTok{"0 (sec)"}\NormalTok{, }\StringTok{"45 (sec)"}\NormalTok{, }\StringTok{"120 (sec)"}\NormalTok{)) }\SpecialCharTok{+}
  \FunctionTok{coord\_flip}\NormalTok{() }\SpecialCharTok{+}
  \FunctionTok{theme\_bw}\NormalTok{()}
\end{Highlighting}
\end{Shaded}

\includegraphics{app-02-flint_files/figure-latex/plot1-1.pdf}

Add labels for \texttt{x}, \texttt{y}, a \texttt{title}, and
\texttt{subtitle} to the code below to update the corresponding plot.

\begin{Shaded}
\begin{Highlighting}[]
\FunctionTok{ggplot}\NormalTok{(}\AttributeTok{data =}\NormalTok{ flint\_focus, }\FunctionTok{aes}\NormalTok{(}\AttributeTok{x =} \FunctionTok{factor}\NormalTok{(zip), }\AttributeTok{y =}\NormalTok{ lead)) }\SpecialCharTok{+}
  \FunctionTok{geom\_boxplot}\NormalTok{(}\FunctionTok{aes}\NormalTok{(}\AttributeTok{fill =} \FunctionTok{factor}\NormalTok{(draw))) }\SpecialCharTok{+} 
  \FunctionTok{labs}\NormalTok{(}\AttributeTok{x =} \StringTok{"Zip Code"}\NormalTok{, }\AttributeTok{y =} \StringTok{"Lead (ppb)"}\NormalTok{, }\AttributeTok{fill =} \StringTok{"Flushing time"}\NormalTok{,}
       \AttributeTok{subtitle =} \StringTok{"Measured in Parts per Billion"}\NormalTok{, }\AttributeTok{title =} \StringTok{"Flint, MI Lead Content by Zip Code"}\NormalTok{) }\SpecialCharTok{+}
  \FunctionTok{scale\_fill\_discrete}\NormalTok{(}\AttributeTok{breaks =} \FunctionTok{c}\NormalTok{(}\StringTok{"first"}\NormalTok{, }\StringTok{"second"}\NormalTok{, }\StringTok{"third"}\NormalTok{),}
                      \AttributeTok{labels =} \FunctionTok{c}\NormalTok{(}\StringTok{"0 (sec)"}\NormalTok{, }\StringTok{"45 (sec)"}\NormalTok{, }\StringTok{"120 (sec)"}\NormalTok{)) }\SpecialCharTok{+}
  \FunctionTok{coord\_flip}\NormalTok{(}\AttributeTok{ylim =} \FunctionTok{c}\NormalTok{(}\DecValTok{0}\NormalTok{, }\DecValTok{50}\NormalTok{)) }\SpecialCharTok{+}
  \FunctionTok{theme\_bw}\NormalTok{()}
\end{Highlighting}
\end{Shaded}

\includegraphics{app-02-flint_files/figure-latex/plot2-1.pdf}

What is the difference between the two plots? While the two plots
contain the same data and graphs, the flipped y-axis (so the x-axis,
ultimately) of the second graph is restricted to the range (0, 50) while
there is no restriction of the range of the first graph beyond the cap
below 1,000 created by the flint\_focus function.

\hypertarget{references}{%
\subsection{References}\label{references}}

\begin{enumerate}
\def\labelenumi{\arabic{enumi}.}
\tightlist
\item
  Langkjaer-Bain, R. (2017). The murky tale of Flint's deceptive water
  data. Significance, 14: 16-21.
\end{enumerate}

\end{document}
