% Options for packages loaded elsewhere
\PassOptionsToPackage{unicode}{hyperref}
\PassOptionsToPackage{hyphens}{url}
%
\documentclass[
]{article}
\title{Data Visualization II}
\author{STA199}
\date{1-18-2022}

\usepackage{amsmath,amssymb}
\usepackage{lmodern}
\usepackage{iftex}
\ifPDFTeX
  \usepackage[T1]{fontenc}
  \usepackage[utf8]{inputenc}
  \usepackage{textcomp} % provide euro and other symbols
\else % if luatex or xetex
  \usepackage{unicode-math}
  \defaultfontfeatures{Scale=MatchLowercase}
  \defaultfontfeatures[\rmfamily]{Ligatures=TeX,Scale=1}
\fi
% Use upquote if available, for straight quotes in verbatim environments
\IfFileExists{upquote.sty}{\usepackage{upquote}}{}
\IfFileExists{microtype.sty}{% use microtype if available
  \usepackage[]{microtype}
  \UseMicrotypeSet[protrusion]{basicmath} % disable protrusion for tt fonts
}{}
\makeatletter
\@ifundefined{KOMAClassName}{% if non-KOMA class
  \IfFileExists{parskip.sty}{%
    \usepackage{parskip}
  }{% else
    \setlength{\parindent}{0pt}
    \setlength{\parskip}{6pt plus 2pt minus 1pt}}
}{% if KOMA class
  \KOMAoptions{parskip=half}}
\makeatother
\usepackage{xcolor}
\IfFileExists{xurl.sty}{\usepackage{xurl}}{} % add URL line breaks if available
\IfFileExists{bookmark.sty}{\usepackage{bookmark}}{\usepackage{hyperref}}
\hypersetup{
  pdftitle={Data Visualization II},
  pdfauthor={STA199},
  hidelinks,
  pdfcreator={LaTeX via pandoc}}
\urlstyle{same} % disable monospaced font for URLs
\usepackage[margin=1in]{geometry}
\usepackage{color}
\usepackage{fancyvrb}
\newcommand{\VerbBar}{|}
\newcommand{\VERB}{\Verb[commandchars=\\\{\}]}
\DefineVerbatimEnvironment{Highlighting}{Verbatim}{commandchars=\\\{\}}
% Add ',fontsize=\small' for more characters per line
\usepackage{framed}
\definecolor{shadecolor}{RGB}{248,248,248}
\newenvironment{Shaded}{\begin{snugshade}}{\end{snugshade}}
\newcommand{\AlertTok}[1]{\textcolor[rgb]{0.94,0.16,0.16}{#1}}
\newcommand{\AnnotationTok}[1]{\textcolor[rgb]{0.56,0.35,0.01}{\textbf{\textit{#1}}}}
\newcommand{\AttributeTok}[1]{\textcolor[rgb]{0.77,0.63,0.00}{#1}}
\newcommand{\BaseNTok}[1]{\textcolor[rgb]{0.00,0.00,0.81}{#1}}
\newcommand{\BuiltInTok}[1]{#1}
\newcommand{\CharTok}[1]{\textcolor[rgb]{0.31,0.60,0.02}{#1}}
\newcommand{\CommentTok}[1]{\textcolor[rgb]{0.56,0.35,0.01}{\textit{#1}}}
\newcommand{\CommentVarTok}[1]{\textcolor[rgb]{0.56,0.35,0.01}{\textbf{\textit{#1}}}}
\newcommand{\ConstantTok}[1]{\textcolor[rgb]{0.00,0.00,0.00}{#1}}
\newcommand{\ControlFlowTok}[1]{\textcolor[rgb]{0.13,0.29,0.53}{\textbf{#1}}}
\newcommand{\DataTypeTok}[1]{\textcolor[rgb]{0.13,0.29,0.53}{#1}}
\newcommand{\DecValTok}[1]{\textcolor[rgb]{0.00,0.00,0.81}{#1}}
\newcommand{\DocumentationTok}[1]{\textcolor[rgb]{0.56,0.35,0.01}{\textbf{\textit{#1}}}}
\newcommand{\ErrorTok}[1]{\textcolor[rgb]{0.64,0.00,0.00}{\textbf{#1}}}
\newcommand{\ExtensionTok}[1]{#1}
\newcommand{\FloatTok}[1]{\textcolor[rgb]{0.00,0.00,0.81}{#1}}
\newcommand{\FunctionTok}[1]{\textcolor[rgb]{0.00,0.00,0.00}{#1}}
\newcommand{\ImportTok}[1]{#1}
\newcommand{\InformationTok}[1]{\textcolor[rgb]{0.56,0.35,0.01}{\textbf{\textit{#1}}}}
\newcommand{\KeywordTok}[1]{\textcolor[rgb]{0.13,0.29,0.53}{\textbf{#1}}}
\newcommand{\NormalTok}[1]{#1}
\newcommand{\OperatorTok}[1]{\textcolor[rgb]{0.81,0.36,0.00}{\textbf{#1}}}
\newcommand{\OtherTok}[1]{\textcolor[rgb]{0.56,0.35,0.01}{#1}}
\newcommand{\PreprocessorTok}[1]{\textcolor[rgb]{0.56,0.35,0.01}{\textit{#1}}}
\newcommand{\RegionMarkerTok}[1]{#1}
\newcommand{\SpecialCharTok}[1]{\textcolor[rgb]{0.00,0.00,0.00}{#1}}
\newcommand{\SpecialStringTok}[1]{\textcolor[rgb]{0.31,0.60,0.02}{#1}}
\newcommand{\StringTok}[1]{\textcolor[rgb]{0.31,0.60,0.02}{#1}}
\newcommand{\VariableTok}[1]{\textcolor[rgb]{0.00,0.00,0.00}{#1}}
\newcommand{\VerbatimStringTok}[1]{\textcolor[rgb]{0.31,0.60,0.02}{#1}}
\newcommand{\WarningTok}[1]{\textcolor[rgb]{0.56,0.35,0.01}{\textbf{\textit{#1}}}}
\usepackage{graphicx}
\makeatletter
\def\maxwidth{\ifdim\Gin@nat@width>\linewidth\linewidth\else\Gin@nat@width\fi}
\def\maxheight{\ifdim\Gin@nat@height>\textheight\textheight\else\Gin@nat@height\fi}
\makeatother
% Scale images if necessary, so that they will not overflow the page
% margins by default, and it is still possible to overwrite the defaults
% using explicit options in \includegraphics[width, height, ...]{}
\setkeys{Gin}{width=\maxwidth,height=\maxheight,keepaspectratio}
% Set default figure placement to htbp
\makeatletter
\def\fps@figure{htbp}
\makeatother
\setlength{\emergencystretch}{3em} % prevent overfull lines
\providecommand{\tightlist}{%
  \setlength{\itemsep}{0pt}\setlength{\parskip}{0pt}}
\setcounter{secnumdepth}{-\maxdimen} % remove section numbering
\ifLuaTeX
  \usepackage{selnolig}  % disable illegal ligatures
\fi

\begin{document}
\maketitle

\hypertarget{main-ideas}{%
\section{Main Ideas}\label{main-ideas}}

\begin{itemize}
\tightlist
\item
  There are different types of variables.
\item
  Visualizations and summaries of variables must be consistent with the
  variable type.
\end{itemize}

\hypertarget{coming-up}{%
\section{Coming Up}\label{coming-up}}

\begin{itemize}
\tightlist
\item
  Lab 2 due tomorrow.
\item
  HW 1 goes out on Thursday.
\end{itemize}

\hypertarget{lecture-notes-and-exercises}{%
\section{Lecture Notes and
Exercises}\label{lecture-notes-and-exercises}}

You will probably not need to do this any more at this stage, but if you
do, please configure git by running the following code in the
\textbf{terminal}. Fill in your GitHub username and the email address
associated with your GitHub account.

\begin{Shaded}
\begin{Highlighting}[]
\NormalTok{git config }\SpecialCharTok{{-}{-}}\NormalTok{global user.name }\StringTok{\textquotesingle{}username\textquotesingle{}}
\NormalTok{git config }\SpecialCharTok{{-}{-}}\NormalTok{global user.email }\StringTok{\textquotesingle{}useremail\textquotesingle{}}
\end{Highlighting}
\end{Shaded}

Next load the \texttt{tidyverse} package. Recall, a package is just a
bundle of shareable code.

\begin{Shaded}
\begin{Highlighting}[]
\FunctionTok{library}\NormalTok{(tidyverse)}
\end{Highlighting}
\end{Shaded}

There are two types of variables \textbf{numeric} and
\textbf{categorical}.

\hypertarget{types-of-variables}{%
\subsubsection{Types of variables}\label{types-of-variables}}

Numerical variables can be classified as either \textbf{continuous} or
\textbf{discrete}. Continuous numeric variables have an infinite number
of values between any two values. Discrete numeric variables have a
countable number of values.

\begin{itemize}
\tightlist
\item
  height
\item
  number of siblings
\end{itemize}

Categorical variables can be classified as either \textbf{nominal} or
\textbf{ordinal}. Ordinal variables have a natural ordering.

\begin{itemize}
\tightlist
\item
  hair color
\item
  education
\end{itemize}

\hypertarget{numeric-variables}{%
\subsubsection{Numeric Variables}\label{numeric-variables}}

To describe the distribution of a numeric we will use the properties
below.

\begin{itemize}
\tightlist
\item
  shape

  \begin{itemize}
  \tightlist
  \item
    skewness: right-skewed, left-skewed, symmetric
  \item
    modality: unimodal, bimodal, multimodal, uniform
  \end{itemize}
\item
  center: mean (\texttt{mean}), median (\texttt{median})
\item
  spread: range (\texttt{range}), standard deviation (\texttt{sd}),
  interquartile range (\texttt{IQR})
\item
  outliers: observations outside the pattern of the data
\end{itemize}

We will continue our investigation of home prices in Minneapolis,
Minnesota.

\begin{Shaded}
\begin{Highlighting}[]
\NormalTok{mn\_homes }\OtherTok{\textless{}{-}} \FunctionTok{read\_csv}\NormalTok{(}\StringTok{"mn\_homes.csv"}\NormalTok{)}
\end{Highlighting}
\end{Shaded}

Add a \texttt{glimpse} to the code chunk below and identify the
following variables as numeric continuous, numeric discrete, categorical
ordinal, or categorical nominal.

\begin{itemize}
\tightlist
\item
  area: numeric continuous
\item
  beds: numeric discrete
\item
  community: categorical nominal
\end{itemize}

\begin{Shaded}
\begin{Highlighting}[]
\FunctionTok{glimpse}\NormalTok{(mn\_homes}\SpecialCharTok{$}\NormalTok{community)}
\end{Highlighting}
\end{Shaded}

\begin{verbatim}
##  chr [1:495] "Calhoun-Isles" "Longfellow" "Longfellow" "Southwest" "Camden" ...
\end{verbatim}

\begin{Shaded}
\begin{Highlighting}[]
\FunctionTok{glimpse}\NormalTok{(mn\_homes}\SpecialCharTok{$}\NormalTok{area)}
\end{Highlighting}
\end{Shaded}

\begin{verbatim}
##  num [1:495] 3937 1440 1835 2016 2004 ...
\end{verbatim}

\begin{Shaded}
\begin{Highlighting}[]
\FunctionTok{glimpse}\NormalTok{(mn\_homes}\SpecialCharTok{$}\NormalTok{beds)}
\end{Highlighting}
\end{Shaded}

\begin{verbatim}
##  num [1:495] 5 2 2 3 3 3 4 3 4 3 ...
\end{verbatim}

The \texttt{summary} command is also useful in looking at numerical
variables. Use this command to look at the numeric variables from the
previous chunk.

\begin{Shaded}
\begin{Highlighting}[]
\FunctionTok{summary}\NormalTok{(mn\_homes}\SpecialCharTok{$}\NormalTok{beds)}
\end{Highlighting}
\end{Shaded}

\begin{verbatim}
##    Min. 1st Qu.  Median    Mean 3rd Qu.    Max. 
##   1.000   3.000   3.000   3.087   4.000   7.000
\end{verbatim}

We can use a \textbf{histogram} to summarize a numeric variable.

\begin{Shaded}
\begin{Highlighting}[]
\FunctionTok{ggplot}\NormalTok{(}\AttributeTok{data =}\NormalTok{ mn\_homes, }
       \AttributeTok{mapping =} \FunctionTok{aes}\NormalTok{(}\AttributeTok{x =}\NormalTok{ salesprice)) }\SpecialCharTok{+} 
   \FunctionTok{geom\_histogram}\NormalTok{(}\AttributeTok{bins =} \DecValTok{25}\NormalTok{)}
\end{Highlighting}
\end{Shaded}

\includegraphics{ae-class-04_files/figure-latex/histogram-1.pdf}

A \textbf{density plot} is another option. We just connect the boxes in
a histogram with a smooth curve.

\begin{Shaded}
\begin{Highlighting}[]
\FunctionTok{ggplot}\NormalTok{(}\AttributeTok{data =}\NormalTok{ mn\_homes, }
       \AttributeTok{mapping =} \FunctionTok{aes}\NormalTok{(}\AttributeTok{x =}\NormalTok{ salesprice)) }\SpecialCharTok{+} 
   \FunctionTok{geom\_density}\NormalTok{()}
\end{Highlighting}
\end{Shaded}

\includegraphics{ae-class-04_files/figure-latex/density-plot-1.pdf}

Side-by-side \textbf{boxplots} are helpful to visualize the distribution
of a numeric variable across the levels of a categorical variable.

\begin{Shaded}
\begin{Highlighting}[]
\FunctionTok{ggplot}\NormalTok{(}\AttributeTok{data =}\NormalTok{ mn\_homes, }
       \AttributeTok{mapping =} \FunctionTok{aes}\NormalTok{(}\AttributeTok{x =}\NormalTok{ community, }\AttributeTok{y =}\NormalTok{ salesprice)) }\SpecialCharTok{+} 
       \FunctionTok{geom\_boxplot}\NormalTok{() }\SpecialCharTok{+} \FunctionTok{coord\_flip}\NormalTok{() }\SpecialCharTok{+} 
       \FunctionTok{labs}\NormalTok{(}\AttributeTok{main=} \StringTok{"Sales Price by Community"}\NormalTok{, }\AttributeTok{x=} \StringTok{"Community"}\NormalTok{, }\AttributeTok{y=}\StringTok{"Sales Price"}\NormalTok{)}
\end{Highlighting}
\end{Shaded}

\includegraphics{ae-class-04_files/figure-latex/boxplots-1.pdf}

\textbf{Question:} What is \texttt{coord\_flip()} doing in the code
chunk above? Try removing it to see. \texttt{coord\_flip()} ensures that
the box plots are laid out horizontally, so they can be compared to a
variable on the x-axis and stacked on top of one another. If it were
absent, the box plots would instead be laid out vertically, side-by-side
with one another and compared to a continuous variable on the y-axis.

\hypertarget{categorical-variables}{%
\subsubsection{Categorical Variables}\label{categorical-variables}}

\textbf{Bar plots} allow us to visualize categorical variables.

\begin{Shaded}
\begin{Highlighting}[]
\FunctionTok{ggplot}\NormalTok{(}\AttributeTok{data =}\NormalTok{ mn\_homes) }\SpecialCharTok{+} 
  \FunctionTok{geom\_bar}\NormalTok{(}\AttributeTok{mapping =} \FunctionTok{aes}\NormalTok{(}\AttributeTok{x =}\NormalTok{ community)) }\SpecialCharTok{+} \FunctionTok{coord\_flip}\NormalTok{() }\SpecialCharTok{+} 
  \FunctionTok{labs}\NormalTok{(}\AttributeTok{main=} \StringTok{"Homes by Community"}\NormalTok{, }\AttributeTok{x=} \StringTok{"Community"}\NormalTok{, }\AttributeTok{y=}\StringTok{"Number of Homes"}\NormalTok{)}
\end{Highlighting}
\end{Shaded}

\includegraphics{ae-class-04_files/figure-latex/bar-plot-1.pdf}

\textbf{Segmented bar plots} can be used to visualize two categorical
variables.

\begin{Shaded}
\begin{Highlighting}[]
\FunctionTok{library}\NormalTok{(viridis)}
\end{Highlighting}
\end{Shaded}

\begin{verbatim}
## Loading required package: viridisLite
\end{verbatim}

\begin{Shaded}
\begin{Highlighting}[]
\FunctionTok{ggplot}\NormalTok{(}\AttributeTok{data =}\NormalTok{ mn\_homes, }\AttributeTok{mapping =} \FunctionTok{aes}\NormalTok{(}\AttributeTok{x =}\NormalTok{ community, }\AttributeTok{fill =}\NormalTok{ fireplace)) }\SpecialCharTok{+} 
  \FunctionTok{geom\_bar}\NormalTok{() }\SpecialCharTok{+}
  \FunctionTok{coord\_flip}\NormalTok{() }\SpecialCharTok{+} 
  \FunctionTok{scale\_fill\_viridis}\NormalTok{(}\AttributeTok{discrete=}\ConstantTok{TRUE}\NormalTok{, }\AttributeTok{option =} \StringTok{"D"}\NormalTok{, }\AttributeTok{name=}\StringTok{"Fireplace?"}\NormalTok{) }\SpecialCharTok{+}
  \FunctionTok{labs}\NormalTok{(}\AttributeTok{main=} \StringTok{"Fireplaces by Community"}\NormalTok{, }\AttributeTok{x=} \StringTok{"Community"}\NormalTok{, }\AttributeTok{y=}\StringTok{"Number of Homes"}\NormalTok{)}
\end{Highlighting}
\end{Shaded}

\includegraphics{ae-class-04_files/figure-latex/segmented-bar-plot-1.pdf}

\begin{Shaded}
\begin{Highlighting}[]
\FunctionTok{ggplot}\NormalTok{(}\AttributeTok{data =}\NormalTok{ mn\_homes, }\AttributeTok{mapping =} \FunctionTok{aes}\NormalTok{(}\AttributeTok{x =}\NormalTok{ community, }\AttributeTok{fill =}\NormalTok{ fireplace)) }\SpecialCharTok{+} 
  \FunctionTok{geom\_bar}\NormalTok{(}\AttributeTok{position =} \StringTok{"fill"}\NormalTok{) }\SpecialCharTok{+} \FunctionTok{coord\_flip}\NormalTok{() }\SpecialCharTok{+} 
  \FunctionTok{scale\_fill\_viridis}\NormalTok{(}\AttributeTok{discrete=}\ConstantTok{TRUE}\NormalTok{, }\AttributeTok{option =} \StringTok{"D"}\NormalTok{, }\AttributeTok{name=}\StringTok{"Fireplace?"}\NormalTok{) }\SpecialCharTok{+}
  \FunctionTok{labs}\NormalTok{(}\AttributeTok{main=} \StringTok{"Percentage of Homes with a Fireplace by Community"}\NormalTok{, }\AttributeTok{x=}
  \StringTok{"Community"}\NormalTok{, }\AttributeTok{y=}\StringTok{"Percentage of Homes"}\NormalTok{)}
\end{Highlighting}
\end{Shaded}

\includegraphics{ae-class-04_files/figure-latex/segmented-bar-plot-fill-1.pdf}

\textbf{Question:} Which of the above two visualizations do you prefer?
Why? Is this answer always the same? I'm not sure if I could say I
``prefer'' one over the other - they show two quite different things. In
the first graph, you get an understanding of how many homes there are in
each community, and within that, how many of them have fireplaces. In
the second graph, your point of comparison is not the number of homes in
each community - which you do not know - but the relative proportion of
homes in each neighborhood that has a fireplace. The second graph
conveys less information, but it does enable much clearer
cross-neighborhood comparison on the basis of whether or not homes have
fireplaces - a worthy tradeoff, if that is your variable of interest.

There is something wrong with each of the plots below. Run the code for
each plot, read the error, then identify and fix the problem.

\begin{Shaded}
\begin{Highlighting}[]
\FunctionTok{ggplot}\NormalTok{(mn\_homes, }\FunctionTok{aes}\NormalTok{(}\AttributeTok{x =}\NormalTok{ lotsize, }\AttributeTok{y =}\NormalTok{ salesprice)) }\SpecialCharTok{+} 
  \FunctionTok{geom\_point}\NormalTok{(}\AttributeTok{shape =} \DecValTok{21}\NormalTok{, }\AttributeTok{size =}\NormalTok{ .}\DecValTok{85}\NormalTok{)}
\FunctionTok{ggplot}\NormalTok{(}\AttributeTok{data =}\NormalTok{ mn\_homes, }\AttributeTok{mapping =} \FunctionTok{aes}\NormalTok{(}\AttributeTok{x =}\NormalTok{ lotsize, }\AttributeTok{y =}\NormalTok{ area)) }\SpecialCharTok{+} 
  \FunctionTok{geom\_point}\NormalTok{(}\AttributeTok{shape =} \DecValTok{21}\NormalTok{, }\AttributeTok{size =}\NormalTok{ .}\DecValTok{85}\NormalTok{)}
\FunctionTok{ggplot}\NormalTok{(}\AttributeTok{data =}\NormalTok{ mn\_homes, }\AttributeTok{mapping =} \FunctionTok{aes}\NormalTok{(}\AttributeTok{x =}\NormalTok{ lotsize, }\AttributeTok{y =}\NormalTok{ area, }\AttributeTok{color=}\NormalTok{community)) }\SpecialCharTok{+}
  \FunctionTok{geom\_point}\NormalTok{(}\AttributeTok{size =} \FloatTok{0.85}\NormalTok{)}
\FunctionTok{ggplot}\NormalTok{(}\AttributeTok{data =}\NormalTok{ mn\_homes, }\AttributeTok{mapping =} \FunctionTok{aes}\NormalTok{(}\AttributeTok{x =}\NormalTok{ lotsize, }\AttributeTok{y =}\NormalTok{ area)) }\SpecialCharTok{+}
  \FunctionTok{geom\_point}\NormalTok{()}
\end{Highlighting}
\end{Shaded}

General principles for effective data visualization

\begin{itemize}
\tightlist
\item
  keep it simple
\item
  use color effectively
\item
  tell a story
\end{itemize}

Why is data visualization important? We will illustrate using the
\texttt{datasaurus\_dozen} data from the \texttt{datasauRus} package.

\begin{Shaded}
\begin{Highlighting}[]
\NormalTok{datasaurus\_dozen }\OtherTok{\textless{}{-}} \FunctionTok{read\_csv}\NormalTok{(}\StringTok{"datasaurus\_dozen.csv"}\NormalTok{)}
\end{Highlighting}
\end{Shaded}

\begin{Shaded}
\begin{Highlighting}[]
\FunctionTok{glimpse}\NormalTok{(datasaurus\_dozen)}
\end{Highlighting}
\end{Shaded}

\begin{verbatim}
## Rows: 1,846
## Columns: 3
## $ dataset <chr> "dino", "dino", "dino", "dino", "dino", "dino", "dino", "dino"~
## $ x       <dbl> 55.3846, 51.5385, 46.1538, 42.8205, 40.7692, 38.7179, 35.6410,~
## $ y       <dbl> 97.1795, 96.0256, 94.4872, 91.4103, 88.3333, 84.8718, 79.8718,~
\end{verbatim}

The code below calculates the correlation, mean of y, mean of x,
standard deviation of y, and standard deviation of x for each of the 13
datasets.

\textbf{Question:} What do you notice? Sneakily, they all have the same
summary statistics - even when they (spoiler alert) aren't necessarily
the same data at all.

\begin{Shaded}
\begin{Highlighting}[]
\NormalTok{datasaurus\_dozen }\SpecialCharTok{\%\textgreater{}\%} 
   \FunctionTok{group\_by}\NormalTok{(dataset) }\SpecialCharTok{\%\textgreater{}\%}
   \FunctionTok{summarize}\NormalTok{(}\AttributeTok{r =} \FunctionTok{cor}\NormalTok{(x, y), }
             \AttributeTok{mean\_y =} \FunctionTok{mean}\NormalTok{(y),}
             \AttributeTok{mean\_x =} \FunctionTok{mean}\NormalTok{(x),}
             \AttributeTok{sd\_x =} \FunctionTok{sd}\NormalTok{(x),}
             \AttributeTok{sd\_y =} \FunctionTok{sd}\NormalTok{(y))}
\end{Highlighting}
\end{Shaded}

\begin{verbatim}
## # A tibble: 13 x 6
##    dataset          r mean_y mean_x  sd_x  sd_y
##    <chr>        <dbl>  <dbl>  <dbl> <dbl> <dbl>
##  1 away       -0.0641   47.8   54.3  16.8  26.9
##  2 bullseye   -0.0686   47.8   54.3  16.8  26.9
##  3 circle     -0.0683   47.8   54.3  16.8  26.9
##  4 dino       -0.0645   47.8   54.3  16.8  26.9
##  5 dots       -0.0603   47.8   54.3  16.8  26.9
##  6 h_lines    -0.0617   47.8   54.3  16.8  26.9
##  7 high_lines -0.0685   47.8   54.3  16.8  26.9
##  8 slant_down -0.0690   47.8   54.3  16.8  26.9
##  9 slant_up   -0.0686   47.8   54.3  16.8  26.9
## 10 star       -0.0630   47.8   54.3  16.8  26.9
## 11 v_lines    -0.0694   47.8   54.3  16.8  26.9
## 12 wide_lines -0.0666   47.8   54.3  16.8  26.9
## 13 x_shape    -0.0656   47.8   54.3  16.8  26.9
\end{verbatim}

Let's visualize the relationships

\begin{Shaded}
\begin{Highlighting}[]
\FunctionTok{ggplot}\NormalTok{(}\AttributeTok{data =}\NormalTok{ datasaurus\_dozen, }
       \AttributeTok{mapping =} \FunctionTok{aes}\NormalTok{(}\AttributeTok{x =}\NormalTok{ x, }\AttributeTok{y =}\NormalTok{ y)) }\SpecialCharTok{+} 
   \FunctionTok{geom\_point}\NormalTok{(}\AttributeTok{size =}\NormalTok{ .}\DecValTok{5}\NormalTok{) }\SpecialCharTok{+} 
   \FunctionTok{facet\_wrap}\NormalTok{( }\SpecialCharTok{\textasciitilde{}}\NormalTok{ dataset)}
\end{Highlighting}
\end{Shaded}

\includegraphics{ae-class-04_files/figure-latex/visualize-dinos-1.pdf}

\textbf{Question:} Why is visualization important? Because data cannot
entirely be defined by its summary statistics - it often has sneaky
explanations lurking below the surface that need to be visualized to be
understood.

\hypertarget{practice}{%
\subsection{Practice}\label{practice}}

\begin{enumerate}
\def\labelenumi{(\arabic{enumi})}
\tightlist
\item
  Modify the code outline to create a faceted histogram examining the
  distribution of year built within each community.
\end{enumerate}

When you are finished, remove \texttt{eval\ =\ FALSE} and knit the file
to see the changes.

\begin{Shaded}
\begin{Highlighting}[]
\FunctionTok{ggplot}\NormalTok{(}\AttributeTok{data =}\NormalTok{ mn\_homes, }\AttributeTok{mapping =} \FunctionTok{aes}\NormalTok{(}\AttributeTok{x =}\NormalTok{ yearbuilt, }\AttributeTok{fill =}\NormalTok{ community)) }\SpecialCharTok{+}
  \FunctionTok{geom\_histogram}\NormalTok{(}\AttributeTok{binwidth =} \DecValTok{10}\NormalTok{) }\SpecialCharTok{+}
  \FunctionTok{facet\_wrap}\NormalTok{(}\SpecialCharTok{\textasciitilde{}}\NormalTok{ community) }\SpecialCharTok{+}
  \FunctionTok{labs}\NormalTok{(}\AttributeTok{x =} \StringTok{"Year Built"}\NormalTok{, }
       \AttributeTok{y =} \StringTok{"Number of Homes"}\NormalTok{,}
      \AttributeTok{title =} \StringTok{"Year Homes Were Built"}\NormalTok{, }
      \AttributeTok{subtitle =} \StringTok{"Faceted by Community in Minneapolis, Minnesota"}\NormalTok{)}\SpecialCharTok{+}
  \FunctionTok{scale\_fill\_viridis}\NormalTok{(}\AttributeTok{discrete =} \ConstantTok{TRUE}\NormalTok{, }\AttributeTok{guide=}\StringTok{"none"}\NormalTok{)}\SpecialCharTok{+}
  \FunctionTok{theme\_bw}\NormalTok{()}
\end{Highlighting}
\end{Shaded}

\includegraphics{ae-class-04_files/figure-latex/ex-1-1.pdf}

\begin{Shaded}
\begin{Highlighting}[]
\FunctionTok{library}\NormalTok{(ggridges)}

\FunctionTok{ggplot}\NormalTok{(mn\_homes, }\FunctionTok{aes}\NormalTok{(}\AttributeTok{x =}\NormalTok{ yearbuilt, }\AttributeTok{y =}\NormalTok{ community, }\AttributeTok{fill =}\NormalTok{ community)) }\SpecialCharTok{+}
  \FunctionTok{geom\_density\_ridges}\NormalTok{(}\AttributeTok{alpha =} \FloatTok{0.3}\NormalTok{, }\AttributeTok{color =} \StringTok{"black"}\NormalTok{, }\AttributeTok{lwd =} \DecValTok{1}\NormalTok{) }\SpecialCharTok{+}
  \FunctionTok{scale\_fill\_viridis}\NormalTok{(}\AttributeTok{discrete =} \ConstantTok{TRUE}\NormalTok{, }\AttributeTok{guide =} \StringTok{"none"}\NormalTok{) }\SpecialCharTok{+}
  \FunctionTok{theme\_bw}\NormalTok{() }\SpecialCharTok{+}
  \FunctionTok{labs}\NormalTok{(}\AttributeTok{x =} \StringTok{"Year Built"}\NormalTok{, }\AttributeTok{y =} \StringTok{"Community"}\NormalTok{, }\AttributeTok{title =} \StringTok{"Year Homes Were Built"}\NormalTok{,}
       \AttributeTok{subtitle =} \StringTok{"Sorted by Community in Minneapolis, Minnesota"}\NormalTok{)}
\end{Highlighting}
\end{Shaded}

\begin{verbatim}
## Picking joint bandwidth of 7.64
\end{verbatim}

\includegraphics{ae-class-04_files/figure-latex/messing_around-1.pdf}

\begin{Shaded}
\begin{Highlighting}[]
\FunctionTok{ggplot}\NormalTok{(mn\_homes, }\FunctionTok{aes}\NormalTok{(}\AttributeTok{x =}\NormalTok{ yearbuilt, }\AttributeTok{y =}\NormalTok{ community, }\AttributeTok{fill =}\NormalTok{ community,}
                     \AttributeTok{height =} \FunctionTok{stat}\NormalTok{(density))) }\SpecialCharTok{+}
  \FunctionTok{geom\_density\_ridges}\NormalTok{(}\AttributeTok{stat =} \StringTok{"binline"}\NormalTok{, }\AttributeTok{binwidth =} \DecValTok{5}\NormalTok{, }\AttributeTok{scale =} \FloatTok{0.99}\NormalTok{,}
                      \AttributeTok{draw\_baseline =} \ConstantTok{FALSE}\NormalTok{) }\SpecialCharTok{+}
  \FunctionTok{scale\_fill\_viridis}\NormalTok{(}\AttributeTok{discrete =} \ConstantTok{TRUE}\NormalTok{, }\AttributeTok{guide =} \StringTok{"none"}\NormalTok{) }\SpecialCharTok{+}
  \FunctionTok{labs}\NormalTok{(}\AttributeTok{x =} \StringTok{"Year Built"}\NormalTok{, }\AttributeTok{y =} \StringTok{"Community"}\NormalTok{, }\AttributeTok{title =} \StringTok{"Year Homes Were Built"}\NormalTok{,}
       \AttributeTok{subtitle =} \StringTok{"Sorted by Community in Minneapolis, Minnesota"}\NormalTok{) }\SpecialCharTok{+}
  \FunctionTok{theme\_bw}\NormalTok{()}
\end{Highlighting}
\end{Shaded}

\includegraphics{ae-class-04_files/figure-latex/messing_around_2-1.pdf}

\hypertarget{additional-resources}{%
\subsubsection{Additional Resources}\label{additional-resources}}

\begin{itemize}
\tightlist
\item
  \url{https://ggplot2.tidyverse.org/}
\item
  \url{https://raw.githubusercontent.com/rstudio/cheatsheets/master/data-visualization-2.1.pdf}
\item
  \url{http://r-statistics.co/Top50-Ggplot2-Visualizations-MasterList-R-Code.html}
\item
  \url{https://medium.com/bbc-visual-and-data-journalism/how-the-bbc-visual-and-data-journalism-team-works-with-graphics-in-r-ed0b35693535}
\item
  \url{https://ggplot2-book.org/} =
  \url{https://ggplot2.tidyverse.org/reference/geom_histogram.html}
\item
  \url{https://rstudio.com/wp-content/uploads/2015/03/ggplot2-cheatsheet.pdf}
\item
  \url{https://github.com/GraphicsPrinciples/CheatSheet/blob/master/NVSCheatSheet.pdf}
\end{itemize}

\end{document}
